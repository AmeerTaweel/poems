\documentclass{article}

% Remove page numbers
\pagenumbering{gobble}

% Support hyperlinks
\usepackage[colorlinks, urlcolor = blue]{hyperref}

% Arabic typesetting
\usepackage{polyglossia}
\setdefaultlanguage{arabic}
\setotherlanguage{english}
\newfontfamily\arabicfont[Script = Arabic]{Amiri}

% Poems typesetting
\usepackage{bidipoem}

\title{لا تصالح}
\author{\textarabic{أمل دنقل}}
\date{}

\begin{document}

% Center horizontally
\centering

% Poem title and author
\maketitle

% Al-Diwan Link
\textenglish{\href{https://www.aldiwan.net/poem25085.html}{https://www.aldiwan.net/poem25085.html}}

% Poem
\begin{modernpoem*}
(\textenglish{1})

لا تصالحْ!
..ولو منحوك الذهب
أترى حين أفقأ عينيك
ثم أثبت جوهرتين مكانهما..
هل ترى..؟
هي أشياء لا تشترى..:
ذكريات الطفولة بين أخيك وبينك،
حسُّكما فجأةً بالرجولةِ،
هذا الحياء الذي يكبت الشوق.. حين تعانقُهُ،
الصمتُ مبتسمين لتأنيب أمكما.. وكأنكما
ما تزالان طفلين!
تلك الطمأنينة الأبدية بينكما:
أنَّ سيفانِ سيفَكَ..
صوتانِ صوتَكَ
أنك إن متَّ:
للبيت ربٌّ
وللطفل أبْ
هل يصير دمي بين عينيك ماءً؟
أتنسى ردائي الملطَّخَ بالدماء..
تلبس فوق دمائي ثيابًا مطرَّزَةً بالقصب؟
إنها الحربُ!
قد تثقل القلبَ..
لكن خلفك عار العرب
لا تصالحْ..
ولا تتوخَّ الهرب!

(\textenglish{2})

لا تصالح على الدم.. حتى بدم!
لا تصالح! ولو قيل رأس برأسٍ
أكلُّ الرؤوس سواءٌ؟
أقلب الغريب كقلب أخيك؟!
أعيناه عينا أخيك؟!
وهل تتساوى يدٌ.. سيفها كان لك
بيدٍ سيفها أثْكَلك؟
سيقولون:
جئناك كي تحقن الدم..
جئناك. كن يا أمير الحكم
سيقولون:
ها نحن أبناء عم.
قل لهم: إنهم لم يراعوا العمومة فيمن هلك
واغرس السيفَ في جبهة الصحراء
إلى أن يجيب العدم
إنني كنت لك
فارسًا،
وأخًا،
وأبًا،
ومَلِك!

(\textenglish{3})

لا تصالح ..
ولو حرمتك الرقاد
صرخاتُ الندامة
وتذكَّر..
(إذا لان قلبك للنسوة اللابسات السواد ولأطفالهن الذين تخاصمهم الابتسامة)
أن بنتَ أخيك "اليمامة"
زهرةٌ تتسربل في سنوات الصبا
بثياب الحداد
كنتُ، إن عدتُ:
تعدو على دَرَجِ القصر،
تمسك ساقيَّ عند نزولي..
فأرفعها وهي ضاحكةٌ
فوق ظهر الجواد
ها هي الآن.. صامتةٌ
حرمتها يدُ الغدر:
من كلمات أبيها،
ارتداءِ الثياب الجديدةِ
من أن يكون لها ذات يوم أخٌ!
من أبٍ يتبسَّم في عرسها..
وتعود إليه إذا الزوجُ أغضبها..
وإذا زارها.. يتسابق أحفادُه نحو أحضانه،
لينالوا الهدايا..
ويلهوا بلحيته (وهو مستسلمٌ)
ويشدُّوا العمامة..
لا تصالح!
فما ذنب تلك اليمامة
لترى العشَّ محترقًا.. فجأةً،
وهي تجلس فوق الرماد؟!

(\textenglish{4})

لا تصالح
ولو توَّجوك بتاج الإمارة
كيف تخطو على جثة ابن أبيكَ..؟
وكيف تصير المليكَ..
على أوجهِ البهجة المستعارة؟
كيف تنظر في يد من صافحوك..
فلا تبصر الدم..
في كل كف؟
إن سهمًا أتاني من الخلف..
سوف يجيئك من ألف خلف
فالدم الآن صار وسامًا وشارة
لا تصالح،
ولو توَّجوك بتاج الإمارة
إن عرشَك: سيفٌ
وسيفك: زيفٌ
إذا لم تزنْ بذؤابته لحظاتِ الشرف
واستطبت الترف

(\textenglish{5})

لا تصالح
ولو قال من مال عند الصدامْ
".. ما بنا طاقة لامتشاق الحسام.."
عندما يملأ الحق قلبك:
تندلع النار إن تتنفَّسْ
ولسانُ الخيانة يخرس
لا تصالح
ولو قيل ما قيل من كلمات السلام
كيف تستنشق الرئتان النسيم المدنَّس؟
كيف تنظر في عيني امرأة..
أنت تعرف أنك لا تستطيع حمايتها؟
كيف تصبح فارسها في الغرام؟
كيف ترجو غدًا.. لوليد ينام
كيف تحلم أو تتغنى بمستقبلٍ لغلام
وهو يكبر بين يديك بقلب مُنكَّس؟
لا تصالح
ولا تقتسم مع من قتلوك الطعام
وارْوِ قلبك بالدم..
واروِ التراب المقدَّس..
واروِ أسلافَكَ الراقدين..
إلى أن تردَّ عليك العظام!

(\textenglish{6})

لا تصالح
ولو ناشدتك القبيلة
باسم حزن "الجليلة"
أن تسوق الدهاءَ
وتُبدي لمن قصدوك القبول
سيقولون:
ها أنت تطلب ثأرًا يطول
فخذ الآن ما تستطيع:
قليلاً من الحق..
في هذه السنوات القليلة
إنه ليس ثأرك وحدك،
لكنه ثأر جيلٍ فجيل
وغدًا..
سوف يولد من يلبس الدرع كاملةً،
يوقد النار شاملةً،
يطلب الثأرَ،
يستولد الحقَّ،
من أَضْلُع المستحيل
لا تصالح
ولو قيل إن التصالح حيلة
إنه الثأرُ
تبهتُ شعلته في الضلوع..
إذا ما توالت عليها الفصول..
ثم تبقى يد العار مرسومة (بأصابعها الخمس)
فوق الجباهِ الذليلة!

(\textenglish{7})

لا تصالحْ، ولو حذَّرتْك النجوم
ورمى لك كهَّانُها بالنبأ..
كنت أغفر لو أنني متُّ..
ما بين خيط الصواب وخيط الخطأ.
لم أكن غازيًا،
لم أكن أتسلل قرب مضاربهم
لم أمد يدًا لثمار الكروم
لم أمد يدًا لثمار الكروم
أرض بستانِهم لم أطأ
لم يصح قاتلي بي: "انتبه"!
كان يمشي معي..
ثم صافحني..
ثم سار قليلاً
ولكنه في الغصون اختبأ!
فجأةً:
ثقبتني قشعريرة بين ضلعين..
واهتزَّ قلبي كفقاعة وانفثأ!
وتحاملتُ، حتى احتملت على ساعديَّ
فرأيتُ: ابن عمي الزنيم
واقفًا يتشفَّى بوجه لئيم
لم يكن في يدي حربةٌ
أو سلاح قديم،
لم يكن غير غيظي الذي يتشكَّى الظمأ

(\textenglish{8})

لا تصالحُ..
إلى أن يعود الوجود لدورته الدائرة:
النجوم.. لميقاتها
والطيور.. لأصواتها
والرمال.. لذراتها
والقتيل لطفلته الناظرة
كل شيء تحطم في لحظة عابرة:
الصبا بهجةُ الأهل صوتُ الحصان التعرفُ بالضيف همهمةُ القلب
حين يرى برعماً في الحديقة يذوي الصلاةُ لكي ينزل المطر الموسميُّ مراوغة القلب حين يرى طائر الموتِ
وهو يرفرف فوق المبارزة الكاسرة
كلُّ شيءٍ تحطَّم في نزوةٍ فاجرة
والذي اغتالني: ليس ربًا..
ليقتلني بمشيئته
ليس أنبل مني.. ليقتلني بسكينته
ليس أمهر مني.. ليقتلني باستدارتِهِ الماكرة
لا تصالحْ
فما الصلح إلا معاهدةٌ بين ندَّينْ..
(في شرف القلب)
لا تُنتقَصْ
والذي اغتالني مَحضُ لصْ
سرق الأرض من بين عينيَّ
والصمت يطلقُ ضحكته الساخرة!

(\textenglish{9})

لا تصالح
ولو وقفت ضد سيفك كل الشيوخ
والرجال التي ملأتها الشروخ
هؤلاء الذين تدلت عمائمهم فوق أعينهم
وسيوفهم العربية قد نسيت سنوات الشموخ
لا تصالح
فليس سوى أن تريد
أنت فارسُ هذا الزمان الوحيد
وسواك.. المسوخ!

(\textenglish{10})

لا تصالحْ
لا تصالحْ
\end{modernpoem*}

\end{document}
